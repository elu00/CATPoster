% !TeX program = lualatex
% https://github.com/anishathalye/gemini

\documentclass[final,noamsthm]{beamer}

% ====================
% Packages
% ====================
\usepackage[T1]{fontenc}
\usepackage{lmodern}
\usepackage[size=custom,width=120,height=72,scale=1]{beamerposter}
\usetheme{gemini}
\usecolortheme{gemini}
\usepackage{graphicx}
\usepackage{booktabs}
\usepackage{tikz}
\usepackage{pgfplots}
\usepackage{multicol}
\usepackage{amsmath}
\usepackage{amsthm}
\setbeamertemplate{theorems}[numbered] % to number
\usepackage{version}

\newtheorem{thm}{Theorem}
\newtheorem{lemma}{Lemma}

\newcommand{\todo}{\\{\Huge{TODO}}\\}
\newcommand{\from}{\colon}
\newcommand{\Z}{\mathbb{Z}}
\newcommand{\Q}{\mathbb{Q}}
\newcommand{\R}{\mathbb{R}}
\newcommand{\F}{\mathbb{F}}
\newcommand{\K}{\mathbb{K}}
\renewcommand{\L}{\mathbb{L}}
\newcommand{\cA}{{\mathcal A}}
\newcommand{\cI}{{\mathcal I}}
\newcommand{\cK}{{\mathcal K}}
\newcommand{\cL}{{\mathscr L}}
\newcommand{\cN}{{\mathcal N}}
\newcommand{\cU}{{\mathcal U}}
\newcommand{\bone}{\mathbbm{1}}
\newcommand{\suchthat}{:}
\renewcommand{\phi}{\varphi}
\newcommand{\actson}{\mathrel{\curvearrowright}}
\newcommand{\normalsg}{\mathrel{\unlhd}}
\newcommand{\symdiff}{\mathrel{\triangle}}
\newcommand{\algcl}[1]{\overline{#1}}
\renewcommand{\th}[1]{{#1}^{\mathrm{th}}}
\newcommand{\opp}[1]{{#1}^{\mathrm{opp}}}
\newcommand{\red}[1]{{\color{red}{#1}}}

\DeclareMathOperator{\powerset}{{\mathcal{P}}}
\DeclareMathOperator{\im}{im}
\DeclareMathOperator{\id}{id}
\DeclareMathOperator{\End}{End}
\DeclareMathOperator{\LT}{LT}
\DeclareMathOperator{\Aut}{Aut}
\DeclareMathOperator{\Hom}{Hom}
\DeclareMathOperator{\PAut}{PAut}
\DeclareMathOperator{\ch}{char}
\DeclareMathOperator{\curl}{curl}
\renewcommand{\div}{\mathbf{div}}
\newcommand{\Diff}{\text{Diff}}
\newcommand{\FDiff}{\text{FDiff}}
\renewcommand{\bar}{\overline}
\def\p{\partial}
\def\ep{\varepsilon}
\def\st{\;\vert\;}
\newcommand{\abs}[1]{\left\vert #1\right\vert}
\newcommand{\norm}[1]{\left\Vert #1\right\Vert}

\newcommand{\colsection}[1]{
    \begin{beamercolorbox}[colsep*=0ex,dp=2ex,center]{block title}
        \vskip0pt
        \usebeamerfont{block section} #1
        \vskip-1.25ex
        \begin{beamercolorbox}[colsep=0.05ex]{block separator}\end{beamercolorbox}
    \end{beamercolorbox}
    {\parskip0pt\par}
    \vskip0pt
}
%\newenvironment{bblock}[1]{\comment}{\endcomment}
\excludeversion{bblock}

% ====================
% Lengths
% ====================

% If you have N columns, choose \sepwidth and \colwidth such that
% (N+1)*\sepwidth + N*\colwidth = \paperwidth
\newlength{\sepwidth}
\newlength{\colwidth}
\setlength{\sepwidth}{0.020\paperwidth}
\setlength{\colwidth}{0.225\paperwidth}

\newcommand{\separatorcolumn}{\begin{column}{\sepwidth}\end{column}}

% ====================
% Title
% ====================
\title{Circular Arc Triangulations}
\author{Ethan Lu}
\institute{Department of Mathematical Sciences, Carnegie Mellon, Pittsburgh}

% ====================
% Footer (optional)
% ====================

\footercontent{
CMU Poster Competition }
% (can be left out to remove footer)

% ====================
% Logo (optional)
% ====================

% use this to include logos on the left and/or right side of the header:
\logoright{\includegraphics[height=7cm]{cmu.png}}
% ====================
% Body
% ====================

\begin{document}

\begin{frame}[t]
    \begin{columns}[t]
        \separatorcolumn

        \begin{column}{\colwidth}
            \colsection{Motivation and Setup}
            \begin{alertblock}{Background/Abstract}
                This project explores how triangulations with circular edges can be a valuable primitive for a variety of geometric algorithms.  An extremely common way to discretize smooth surfaces is using Euclidean triangles, with straight edges---however, such triangles do a poor job of capturing some important features of smooth surfaces, such as its curvature (which becomes highly singular), and angle-preserving or \emph{conformal} maps (which cannot preserve angles in the discrete setting).  On the other hand, surface approximations using higher-order patches (NURBS, etc.) become difficult to control, and to reason about.  Circular arc triangulations (CATs) provide a middle ground, where only a few additional degrees of freedom give a significant geometric improvement.  We describe a variety of ways of parameterizing CATs, including a novel parameterization by a triangle in a constant-curvature geometry, plus a M\"{o}bius transformation of the plane---different parameterizations in turn allow us to formulate efficient algorithms and optimization problems in various settings.  In particular, we show how to approximate a curved surface by a CAT mesh with curvature only along edges (rather than at vertices); using a connection to \emph{circle patterns} we describe a discrete version of Riemann mapping where angles are directly preserved; and, X.
            \end{alertblock}

            \begin{block}{Setup and Notation}
                \begin{center}
                    \includegraphics[width = 0.8\textwidth] {flowmap.pdf}\\
                \end{center}
                Throughout, our domain of interest will be a fixed open and connected subset of $\R^n$, which will serve as an initial reference frame for our fluid.
                Denoting this set by $\Omega$, we then set $\Sigma := \partial \Omega$ to be it's boundary and $\nu:\Sigma\to\R^n$ to be the associated outward pointing unit normal.

                Given any such domain, we can then define the function spaces $\Diff_0(\Omega),\FDiff(\Omega) \subseteq L^2(\Omega;\R^n)$, which are set to be
                \begin{align*}
                    \Diff_0(\Omega) & =\{\eta: \Omega \to \Omega\st  \eta \in \FDiff(\Omega)\}                                   \\
                    \FDiff(\Omega)  & =\{\eta :\Omega \to \R^n\st\eta \text{ a volume/orientation preserving diffeomorphism}\} .
                \end{align*}
            \end{block}





        \end{column}
        \separatorcolumn
        \begin{column}{\colwidth}
            \colsection{Technical Tools}
            \begin{block}{Existence of a Perturbation}
                With these preliminaries established, we'll now move onto various technical results, which will facilitate the calculations and constructions used to derive our final PDEs.
                To begin, consider the space $X$ of all flows associated to $\Omega$ over the time interval $[0,1]$; that is,
                \[
                    X:=\{\eta\in C^1([0,1];\FDiff(\Omega))\st \eta(0)=\eta_0,\eta(1)=\eta_1\}
                \]
                where $\eta_0, \eta_1$ are some fixed initial and terminal states of the fluid.
                Given any $\eta \in X$, we can then view this function as a path between $\eta_0$ and $\eta_1$ along the manifold $\FDiff$, with $\eta(t)$ encoding ``where the fluid is at time $t$.''
                Since we're then interested in characterizing \emph{shortest} paths along $\FDiff$, a natural question to then ask would be what local changes can be made to such a path $\eta$. \\
                Fixing any such one-parameter family of flows $\zeta(s):(-\ep,\ep)\to X$ with $\zeta(0) = \eta$, an obvious necessary condition is that $\partial_s \zeta(0)(t)$ is in the tangent space of $\FDiff$ at $\eta(t)$.
                Using techniques from the theory of ODE,  we can show that this is also a sufficient condition; that is, given an arbitrary velocity field $v_0$, we can find $\zeta$ parametrizing perturbations of $\eta$ such that the derivative of $\zeta$ at $\eta$ is equal to the desired velocity field.

                Formally, the statement is as follows.
                \begin{lemma}
                    Let $v_0:[0,1]\to \{v\in L^2(\Omega; \R^n) \st \div\;(v\circ\eta^{-1})=0\}$, $\eta_0,\eta_1\in \FDiff(\Omega)$ be fixed.\\
                    Then there exists a perturbation $ \zeta: (-\ep, \ep) \to X$ such that:
                    \[
                        \zeta(0)=\eta ,\zeta(s)\in C^\infty, \text{ and }
                        \p_s \zeta(x,0,t):=v(\eta(x,t),0,t)=v_0(\eta(x,t),t)
                    \]
                \end{lemma}
                Using this lemma, since we now know that $\frac{d\zeta}{ds}$ can be specified arbitrarily, we obtain useful characterizations of local minima of the energies that we can use in conjunction with the following results.
                %This will allow us to see that if a function integrated against arbitrary $\frac{d\zeta}{ds}$ is zero, then the function must be 0.
            \end{block}



            \begin{block}{Reynolds Transport Theorem on Hypersurfaces}
                The final result we will need is the Reynolds Transport Theorem, which is a result that allows us to obtain explicit formulae for the derivatives of boundary terms in our energies, which we will need for those energies involving surfactants.
                \begin{thm}
                    Let $\Sigma$ be a hypersurface and $\beta\in C^1(\Sigma\times [0,1]; \R^n)$. Set $\Sigma(t)=\beta(\Sigma,t)$. If $f\in C^1(\R^n\times [0,1];\R^n)$, then
                    \begin{equation}
                        \dfrac{d}{dt}\int_{\Sigma(t)}f=\int_{\Sigma(t)}\p_t f+\nabla f\cdot u+f\div_{\Sigma(t)}u
                    \end{equation}
                    where $u(\beta(x,t),t)=\p_t\beta(x,t)$ and the surface divergence is
                    $
                    \div_{\Sigma(t)}u=\text{tr}((I-\nu\otimes\nu)Du).
                    $
                \end{thm}
            \end{block}


        \end{column}



        \separatorcolumn

        \begin{column}{\colwidth}
            \colsection{Main Results}
            \begin{block}{Previous results: Arnold's setup }
                We start with Arnold's original derivation of the Euler equations, noting in particular the relative simplicity of the associated energy.
                \begin{thm}[Arnold]
                \end{thm}
                \begin{proof}[Proof Sketch]
                \end{proof}

            \end{block}
            \begin{block}{Result \texttt{\#}1: Surface tension and potential}
                We now consider a significant complication of the Arnold functional, where we introduce a globally defined potential term $\varphi$ (which can represent forces such as gravity or electromagnetism), allow the Eulerian and densities $\rho$ of the fluid to vary over space, add a term $\sigma$ to compensate for surface tension, and allow the fluid to move freely through space.
                \begin{thm}
                    Given a constant $\sigma\in \R^+$, $\bar\rho:\Omega\to \R^+$ and $\varphi\in C^1(\R^n)$, minimizers (if they exist) of the energy $E:X\to \R$ defined via
                    \begin{equation}
                        E(\eta)=\int_0^1\left(\int_{\Omega}\dfrac{\bar \rho }{2}|\p_t\eta|^2-\varphi(\eta)\;dx-\int_{\p\Omega(t)}\sigma dS\right)dt
                    \end{equation}
                    must satisfy the incompressible Euler equations with surface tension; that is,
                    \begin{equation}
                        \begin{cases}
                            \rho (\p_t u+u\cdot\nabla u)+\nabla p=-\nabla \varphi & \text{ on }\Omega(t)   \\
                            \div\; u= 0                                           & \text{ on }\Omega(t)   \\
                            p=-\sigma H                                           & \text{ on }\p\Omega(t)
                        \end{cases}
                    \end{equation}
                    where $\Omega(t) := \eta (\Omega, t)$, $u$ is the Eulerian velocity defined via $u(\eta(x,t),t)=\p_t\eta(x,t)$, $\bar \rho,\rho$ are Lagrangian and Eulerian  densities, and $H=-\div\;\nu$ is the mean curvature of $\p\Omega(t)$.
                \end{thm}
                \begin{proof}[Proof Sketch]
                    We proceed as in the previous result.
                    By first only considering compactly supported velocity fields, we can isolate the contribution of the terms defined on $\Omega$ to deduce the first equation.
                    Considering general velocity fields, combining the Reynolds transport equation and the surface divergence theorem, and doing further computations then yields the other equations.
                \end{proof}

            \end{block}

            %\colsection{Results (contd)}








        \end{column}

        \separatorcolumn

        \begin{column}{\colwidth}
            \vspace{-1.5cm}

            \begin{block} {Result \texttt{\#}3: Surfactants freely moving along boundary}
                Finally, we introduce another degree of freedom by allowing the surfactants to move along the boundary independently of the mass in the interior, as parameterized by a function $\beta$.
                \begin{thm}
                    Let $X$ be as above and
                    \begin{equation*}
                        Y=\{\beta\in C^1([0,1];\Diff_0(\p\Omega))\st \beta(0,x)=x\}
                    \end{equation*}
                    Given $\bar \rho:\Omega\to \R^+,\ \xi:\R\to \R^+, \ \bar\gamma_0:\p\Omega\to \R^+,\ \varphi\in C^1(\R^n)$, consider $E:X\times Y\to \R$ via
                    \begin{equation}
                        E(\eta,\beta)=\int_0^1\left(\int_{\Omega}\dfrac{\bar \rho}{2}|\p_t\eta|^2-\varphi(\eta)\;dx+\int_{\p\Omega}\dfrac{\bar\gamma_0}{2}|\p_t(\eta\circ\beta)|^2\;dS-\int_{\p\Omega(t)}\xi(\gamma)\;dS\right)dt
                    \end{equation}
                    with all relevant terms as defined above.

                    Then minimizers (if they exist) of the energy functional $E$ must satisfy
                    \begin{equation}
                        \label{final_system}
                        \begin{cases}
                            \rho(\p_t u+u\cdot \nabla u) +\nabla p=-\nabla \varphi                         & \text{on }\Omega(t)   \\
                            \div\;u=0                                                                      & \text{on }\Omega(t)   \\
                            \gamma(\p_t u_s+u_s\cdot \nabla u_s)-p\nu=\nabla_{\Sigma(t)}\sigma+H\nu \sigma & \text{on }\p\Omega(t) \\
                            \p_t\gamma+\nabla\gamma\cdot u+\gamma\div_{\Sigma(t)}u=0                       & \text{on }\p\Omega(t)
                        \end{cases}
                    \end{equation}
                    where $u_s$ is the Eulerian velocity of the surfactants (defined via $u_s(\eta(\beta(x,t),t),t)=\p_t (\eta(\beta(x,t),t))$), $\sigma=\xi(\gamma)-\xi'(\gamma)\gamma$ is the surface tension, and $u,\bar \rho,\rho,$ and $p$ are as before.
                \end{thm}


                Note that while this result and the previous result appear to be very similar, the surfactant velocity $u_s$ is now generated not only by the motion of $\eta$ but also by the motion of the surfactants $\beta$, which complicates the corresponding terms. %while $u$ depends solely upon $\eta$.
            \end{block}

            \begin{block}{Acknowledgements}
                This project was joint work with Jonathan Jenkins and Desmond Reed.
                We would also like to thank advisor Ian Tice for his guidance and mentorship throughout this project. This project was supported by funding from the NSF CAREER grant (\texttt{\#}1653161).
            \end{block}
            \begin{block}{References}
                \vspace{-0.3em}
                {\tiny
                    \nocite{*}
                    \begin{multicols}{2}
                        \bibliographystyle{abbrv}
                        \bibliography{refs}
                \end{multicols}}
            \end{block}
        \end{column}
        \separatorcolumn
    \end{columns}
\end{frame}

\end{document}

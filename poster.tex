% !TeX program = lualatex
% https://github.com/anishathalye/gemini

\documentclass[final,noamsthm]{beamer}

% ====================
% Packages
% ====================
\usepackage[T1]{fontenc}
\usepackage{lmodern}
\usepackage[size=custom,width=120,height=72,scale=1]{beamerposter}
\usetheme{gemini}
\usecolortheme{mit}
\usepackage{graphicx}
\usepackage{booktabs}
\usepackage{tikz,tikz-cd}
\usepackage{pgfplots}
\usepackage{multicol}
\usepackage{amsmath}
\usepackage{amsthm}
\setbeamertemplate{theorems}[numbered] % to number
\usepackage{version}

\newtheorem{thm}{Theorem}
\newtheorem{ex}{Example}
\newtheorem{defn}[ex]{Definition}
\newtheorem{lemma}[ex]{Lemma}

\newcommand{\todo}{\\{\Huge{TODO}}\\}
\newcommand{\from}{\colon}
\newcommand{\Z}{\mathbb{Z}}
\newcommand{\N}{\mathbb{N}}
\newcommand{\Q}{\mathbb{Q}}
\newcommand{\R}{\mathbb{R}}
\newcommand{\F}{\mathbb{F}}
\newcommand{\K}{\mathbb{K}}
\renewcommand{\L}{\mathbb{L}}
\newcommand{\cA}{{\mathcal A}}
\newcommand{\cI}{{\mathcal I}}
\newcommand{\cK}{{\mathcal K}}
\newcommand{\cL}{{\mathscr L}}
\newcommand{\cN}{{\mathcal N}}
\newcommand{\cU}{{\mathcal U}}
\newcommand{\bone}{\mathbbm{1}}
\newcommand{\suchthat}{:}
\renewcommand{\phi}{\varphi}
\newcommand{\actson}{\mathrel{\curvearrowright}}
\newcommand{\normalsg}{\mathrel{\unlhd}}
\newcommand{\symdiff}{\mathrel{\triangle}}
\newcommand{\algcl}[1]{\overline{#1}}
\renewcommand{\th}[1]{{#1}^{\mathrm{th}}}
\newcommand{\opp}[1]{{#1}^{\mathrm{opp}}}
\newcommand{\red}[1]{{\color{red}{#1}}}

\DeclareMathOperator{\powerset}{{\mathcal{P}}}
\DeclareMathOperator{\diam}{{diam}}
\DeclareMathOperator{\Unif}{{Unif}}
\DeclareMathOperator{\im}{im}
\DeclareMathOperator{\id}{id}
\DeclareMathOperator{\End}{End}
\DeclareMathOperator{\LT}{LT}
\DeclareMathOperator{\Aut}{Aut}
\DeclareMathOperator{\Hom}{Hom}
\DeclareMathOperator{\PAut}{PAut}
\DeclareMathOperator{\ch}{char}
\DeclareMathOperator{\curl}{curl}
\renewcommand{\div}{\mathbf{div}}
\newcommand{\Diff}{\text{Diff}}
\newcommand{\FDiff}{\text{FDiff}}
\renewcommand{\bar}{\overline}
\def\p{\partial}
\def\ep{\varepsilon}
\def\st{\;\vert\;}
\newcommand{\abs}[1]{\left\vert #1\right\vert}
\newcommand{\norm}[1]{\left\Vert #1\right\Vert}

\newcommand{\colsection}[1]{
    \begin{beamercolorbox}[colsep*=0ex,dp=2ex,center]{block title}
        \vskip0pt
        \usebeamerfont{block section} #1
        \vskip-1.25ex
        \begin{beamercolorbox}[colsep=0.05ex]{block separator}\end{beamercolorbox}
    \end{beamercolorbox}
    {\parskip0pt\par}
    \vskip0pt
}
%\newenvironment{bblock}[1]{\comment}{\endcomment}
\excludeversion{bblock}

% ====================
% Lengths
% ====================

% If you have N columns, choose \sepwidth and \colwidth such that
% (N+1)*\sepwidth + N*\colwidth = \paperwidth
\newlength{\sepwidth}
\newlength{\colwidth}
\setlength{\sepwidth}{0.020\paperwidth}
\setlength{\colwidth}{0.225\paperwidth}

\newcommand{\separatorcolumn}{\begin{column}{\sepwidth}\end{column}}

% ====================
% Title
% ====================
\title{A Probabilistic Analysis of Enhanced Dissipation}
\author{Ethan Lu}
\institute{Department of Mathematical Sciences, Carnegie Mellon, Pittsburgh}

% ====================
% Footer (optional)
% ====================

\footercontent{
CMU Poster Competition }
% (can be left out to remove footer)

% ====================
% Logo (optional)
% ====================

% use this to include logos on the left and/or right side of the header:
\logoright{\includegraphics[height=7cm]{cmu.png}}
% ====================
% Body
% ====================

\begin{document}

\begin{frame}[t]
    \begin{columns}[t]
        \separatorcolumn

        \begin{column}{\colwidth}
            \colsection{Motivation and Setup}
            \begin{alertblock}{Background/Abstract}
This project concerns itself with the phenomenon of \textbf{enhanced dissipation}, an effect that commonly manifests itself in a variety of physical contexts, ranging from tumor growth and bacterial movement to fluid flow and chemotaxis (through e.g., the Cahn-Hilliard and Keller-Segel equations).
The basic setup can be thought of as follows. Suppose you’re given a container of incompressible fluid and a drop of dye and are tasked with coloring all of the water as quickly as possible.
While you could just drop in the dye and step away, perhaps the more natural thing to do would be to put the dye in and then to stir the fluid, thereby speeding up the process of convection.
This simple observation captures exactly the idea of \emph{enhanced dissipation}, which has become a recent subject of study within the PDE community (see \cite{Thiffeault_2012} for a review).
Although precise formulations of this phenomenon vary significantly from problem to problem, perhaps one of the simplest models to consider is that of the advection–diffusion equation, which reads 
\begin{align}
    \p_t \theta + u \cdot \nabla \theta - \kappa \Delta \theta = 0
\end{align}
where $\theta$ is understood to represent the concentration of a dye, $u$ is a (possibly time-dependent and divergence free) velocity field advecting the fluid, and $\kappa$ is the diffusivity of the dye.
Within this context, a standard measure of dissipation is to consider bounds on the \emph{variance} of such solutions (e.g., bounds on $\norm{ \theta- \int- \theta}_{L^2}$).
Using a simple energy estimate, we can obtain the bound
\[
        \norm{\theta(\cdot,t)}_{L^2_0(\Omega)} \le 
        e^{- \lambda_1 \kappa t}\norm{\theta(\cdot,0)}_{L^2_0(\Omega)} 
\]
for all mean 0 solutions to the equations above, where $\lambda_1$ denotes the smallest non-zero eigenvalue of the Laplacian.
Although such a calculation already yields a useful insight into the nature of the problem, the energy estimate above is inherently limited because it fails to consider advection. Indeed, after integrating by parts, the effect of $u$ cancels, meaning that this bound has no dependence on the bulk motion of the fluid.  
The primary question to answer, then, is to see how different properties of $u$ can be parlayed into improved bounds on the exponent given above, also known as the \emph{dissipation time} of the system.
Towards doing so, the primary relation we will be interested in studying is that between the \emph{ergodicity} of such a system and its dissipation time. %the exact definitions of which will be quantified later. 

            \end{alertblock}

            \begin{block}{Setup and Notation}
                Throughout, our domain of interest will be a fixed smooth Riemannian manifold $M$, which we'll typically assume to just be given by the $n$ dimensional torus, $\mathbb T^n := (\R/\Z)^n$. We'll then equip this space with it's natural measure $\pi$, normalized so that $M$ has total mass $1$.

                Given any such domain, we can then define the function spaces $L^2(M),L^2_0(M), H^s(M), \dot H^\alpha(M)$, which are the spaces of square integrable functions, square integrable functions with mean 0, the usual Sobolev space of order $s$, and the homogenous Sobolev space of order $\alpha$.
            \end{block}





        \end{column}
        \separatorcolumn
        \begin{column}{\colwidth}
            \colsection{Continuous Time}
            \begin{block}{Previous Work}
    In continuous time, recent results from \cite{Feng_2019,Zelati_Delgadino_Elgindi_2020} have shown that the naive $\mathcal O( \kappa^{-1})$ bound on the dissipation time can be improved to $\mathcal O(\ln \kappa)^2$ given an ``exponential mixing'' assumption on the underlying flow $u$, the details of which are the focus of this section.
While already a substantial improvement to the naive bounds discussed above, however, there still remains the natural question of whether or not these bounds are optimal.
Heuristically, the answer is no: in addition to intuitive arguments suggesting an optimal bound of order $\mathcal{O}({\ln \kappa})$, several classes of exponentially mixing flows for which this bound holds have been explicitly constructed, suggesting that it could hold in the general case (see \cite{Feng_2019}).

                \begin{thm}
    Suppose that $u$ is a smooth, divergence free, and exponentially mixing velocity field.
    Then for $\kappa \ll 1$, the dissipation time of the associated advection-diffusion equations is bounded by
    \[
        \tau_d \le  C\abs{\ln{\kappa}}^2
    \]
    for some $C > 0$.
                \end{thm}

                In order to prove this result, we will naturally need a number of technical results,which we present now.
            \end{block}


\begin{block}{Technical Results}
\begin{defn}
		The \textbf{dissipation time} associated to the flow generated by $u, \kappa$ is
		\[
			\tau_d := \inf \{t-s \st \norm{\mathcal{S}{s,t}(\theta_s)} \le \dfrac{\norm {\theta_s}}{e} \quad \forall \theta_s \in L^2_0, t \in \R\}.
		\]
	\end{defn}
	\begin{lemma}[Energy Equality]
		For any $s \le t$, we have that
		\[
			\norm{\theta(t)}^2
			=\norm{\theta(s)}^2 \exp \left(-2 \kappa \int_s^t\dfrac{\norm{\theta(r)}^2_{\dot H^1}}{\norm{\theta(r)}^2} dr\right).
		\]
	\end{lemma}
	\begin{lemma}
		Let $\phi$ be the solution to the system above with $\kappa = 0$.
		Then
		\begin{align*}
			\norm{\theta(t)-\phi(t)}^{2} \leqslant 2 \sqrt{2 \kappa(t-s)}\norm{\theta_{s}}\left(2\|\nabla u\|_{L^{\infty}} \int_{s}^{t}\norm{\theta}_{H^{1}}^{2} +\left\|\theta_{s}\right\|_{H^{1}}^{2}\right)^{1 / 2} .
		\end{align*}
	\end{lemma}
	\begin{lemma}
		Let $H(\kappa) \in \R$ satisfy
		\[
			\sqrt{H(\kappa)} (\ln c_1 + \ln(H(\kappa)) + \ln 2) = \dfrac{c_2}{64 \sqrt{\kappa\norm{\nabla u}_{L^\infty}}}
		\]
		and suppose that $\lambda_N$ is the largest eigenvalue of $\Delta$ in $[0,H(\kappa)]$.
		Then if
		\(
		\norm{\theta_{s}}_{\dot H^1}^{2}<\lambda_{N}\norm{\theta_{s}}^{2}
		\)
		we have the estimate
		\[
			\norm{\theta\left(s+ t_0\right)}^{2} \le\exp \left(-\frac{\kappa H(\kappa) t_0}{8}\right)\norm{\theta_{s}}^{2}
		\]
		where
		\[
			t_0 := 2\left(\ln c_1+\ln\left({2\lambda_N}\right)\right)/{c_2}.
		\]
	\end{lemma}



            \end{block}


            
        \end{column}



        \separatorcolumn
        \begin{column}{\colwidth}
            \colsection{Discrete Time}
            \begin{block}{Motivation and Examples}
                In continuous time, we considered a model of our underlying system as a smooth density continuously affected by bulk movement (advection) and small scale scattering (diffusion).
            In discrete time, we'll be considering a slightly different model of the underlying dynamics, splitting the advection and diffusion terms into two operators represented by interleaving a dynamical system and Markov Kernel.
                %Motivation: Fundamental solution of the heat equation, convergence of random walks with drift.
\begin{ex}[Doubling Map]
    Let $\mathbb{T} := \R / \Z$, and $T: \mathbb T \to \mathbb T$ be given by $x \mapsto 2x$.
    For $\kappa > 0$, let $K_\kappa$ be the transition kernel corresponding to the operator $e^{ \kappa \Delta}$, and consider the Markov transition kernel given by $TK_\kappa$.
    Then the Lebesgue measure is the unique invariant measure of this system with $\tau_{mix} \le C \abs{\ln \kappa}$.
\end{ex}
\begin{ex}[Doubling Map with Uniform Noise]
   Let $\mathbb{T}, T$ be as in the previous theorem, and for $n \in \N$, let $K_n$ be the transition kernel given by 
   \[
        K_n(x, A) \sim \pi(A \cap [x-2^{-N},x +2^{-N}])
   \]
   where again $\pi$ denotes the Lebesgue measure.
   Then the Markov transition kernel given by $TK_\kappa$ has the Lebesgue measure as its the nique invariant measure with $\tau_{mix} \le 2n$.
\end{ex}
\begin{ex}[Baker's Map]
Let $M := \mathbb{T}^2$ and $T:\mathbb{T}^2 \to \mathbb{T}^2$ be the dynamical system given by
\[
    (0.x_1x_2\cdots, 0.y_1y_2\cdots) \mapsto (0.x_2\cdots, 0.x_1y_1y_2\cdots),
\] 
where $0.x_1x_2\cdots$ denotes the binary expansion of $x\in [0,1)$. \\
For $N \in \N$, let $K_N$ be the transition kernel associated to the process 
\[
    X_{i+1} = X_i + 2^{-N}\Unif[-1,1]
\]
Then the Lebesgue measure is the unique invariant measure of the kernel $T P_N$ with associated mixing time $\tau_{mix} \le C N$.
\end{ex}



            \end{block}
            \begin{block}{Markov Processes and Dynamical Systems}
                Recall that a \textbf{Markov transition kernel} $K$ on $M$ is a function $K:M \times \mathcal{B}(M) \to [0,1]$ such that
        %\item For any fixed $A \in \b(M)$, $K(\cdot, A)$ is measurable, 
        $K(x,\cdot)$ is a probability measure on $M$ for every $x \in M$, with a natural extension to a linear map between signed measures via the action
    \[
        K\star \mu(A) := \int_{M}K(x,A) \mu(dx).
    \]
    \begin{defn}
        A series of $M$-valued random variables $\{X_i\}_{i \in \N}$ is a \textbf{Markov process with kernel} $K$ if $X_{i+1}$ is independent of $X_j$ for all $j < i$ conditioned on $X_i$, and $\mu_{X_{i+1}} = K \mu_{X_i}$ for all $i \in \N$.
    \end{defn}
    \begin{defn}
    Given a Markov kernel $K$ with unique stationary measure $\pi$, we define the \textbf{mixing time} $\tau_{mix}$ of $K$ to be
    \[
        \tau_{mix} :=\sup_{\mu }\inf_{n \in \N} \norm{K^n \mu - \pi}_{TV} \le 1/e.
    \]
    where the supremum is taken over all probability measures $\mu$.
 
    \end{defn}
            \end{block}

        \end{column}

        \separatorcolumn

        \begin{column}{\colwidth}
            \vspace{-1.5cm}

            \begin{block}{Main Result}
                We can now state our main result.
                Recall that we say $\varphi$ is \emph{Bernoulli} if there exists a Bernoulli shift $(\Omega, \mathcal B, \mu)$ and a bijective, measure preserving $\pi \colon M \to \Omega$ such that
  \begin{equation*}
    \pi \circ \varphi = S \circ \pi\,,
  \end{equation*}
  where $S$ is the shift operator on~$\Omega$.
  In other words, we have that the following diagram commutes:
%  \tikzset{commutative diagrams/.cd,
%smoffset/.style={start anchor=center,end anchor=center,draw=none,yshift=-5pt,xshift=-8pt}
%}

%\newcommand\commsmile[2][\smiley]{\arrow[smoffset]{#2}{#1}}
    \[
        \begin{tikzcd}[row sep=large,column sep=huge,ampersand replacement=\&]
        M  \ar[d, "\pi"] \ar[r, "\varphi"]\& M \ar[d, "\pi"]\\
        \Omega \ar[r, "S"] \& \Omega
        \end{tikzcd}
    \]
\begin{defn}
  Given a manifold $M$ and a map $\pi: M \to \Omega$ as above, we say that the Bernoulli map $\varphi$ is exponentially mixing
    \[
        \sup_{x \in M} \diam(A_{-N,N,x}) \le c_1 e^{-c_2 N}
    \forall N \in \N.
    \]
\end{defn}


                \begin{thm}
Let $M$ be a Riemannian manifold and $\varphi: M \to M$ be a smooth exponentially mixing dynamical system (in the sense above).
			Suppose further that $K_\kappa$ is a Markov transition kernel on $M$ such that
            \begin{enumerate}
        \item on $M$, $K$ satisfies
      \begin{equation*}
        K(x, \cdot) \geq \alpha \frac{1}{\mu{B(x, \kappa)}} \chi_{B(x, \kappa)} d \mu
      \end{equation*}
      for some $\alpha > 0$.%,  $N$ with $h(N) \le \kappa/2$
        \item \textbf{[Homogeneity]} For all $-N \le m \le N$, there exists a bijective $M_{m,\omega, \eta}: \Omega \to \Omega$ such that if $\omega_i = \eta_i$ for all $i \ge m$,
            \[
                M_{m,\omega, \eta} \star K(\omega,\cdot) = K(\eta, \cdot)
            \]
            and $M(\gamma)_i = \gamma_i$ for all $i \ge m, \gamma \in \Omega$.
       \end{enumerate}
			Then the mixing time of the Markov process associated to $\varphi \star K_\kappa$ satisfies the estimate
			\[
				\tau_{mix} \le C\abs{\ln \kappa}.
			\]
            \begin{proof}[Proof Sketch]
                By standard results from probability, it suffices to construct a coupling of any two initial distributions that couples in expected $\mathcal{O}(N)$ time.
                To do so, we begin by using the first assumption to put any two initial distributions in the same ``relative position'' from the perspective of the Bernoulli system.
                After this, using homogeneity, we can force any two points to be within distance $\kappa$ of each other in order $N$ steps, after which we can make them couple exactly in order 1 time.
            \end{proof}

                \end{thm}
            \end{block}

            %\begin{block}{Acknowledgements}
            %    Thanks 
                %This project was joint work with Jonathan Jenkins and Desmond Reed.
                %We would also like to thank advisor Ian Tice for his guidance and mentorship throughout this project. This project was supported by funding from the NSF CAREER grant (\texttt{\#}1653161).
            %\end{block}
            \begin{block}{References}
                \vspace{-0.3em}
                {
                    \tiny
                    \nocite{*}
                    \begin{multicols}{2}
                        \bibliographystyle{abbrv}
                        \bibliography{refs}
                \end{multicols}}
            \end{block}
        \end{column}
        \separatorcolumn
    \end{columns}
\end{frame}

\end{document}

% !TeX program = lualatex
% https://github.com/anishathalye/gemini

\documentclass[final,noamsthm]{beamer}

% ====================
% Packages
% ====================
\usepackage[T1]{fontenc}
\usepackage{lmodern}
\usepackage[size=custom,width=120,height=72,scale=1]{beamerposter}
\usetheme{gemini}
\usecolortheme{mit}
\usepackage{graphicx}
\usepackage{booktabs}
\usepackage{tikz}
\usepackage{pgfplots}
\usepackage{multicol}
\usepackage{amsmath}
\usepackage{amsthm}
\setbeamertemplate{theorems}[numbered] % to number
\usepackage{version}

\newtheorem{thm}{Theorem}
\newtheorem{defn}{Definition}
\newtheorem{lemma}{Lemma}

\newcommand{\todo}{\\{\Huge{TODO}}\\}
\newcommand{\from}{\colon}
\newcommand{\Z}{\mathbb{Z}}
\newcommand{\Q}{\mathbb{Q}}
\newcommand{\R}{\mathbb{R}}
\newcommand{\F}{\mathbb{F}}
\newcommand{\K}{\mathbb{K}}
\renewcommand{\L}{\mathbb{L}}
\newcommand{\cA}{{\mathcal A}}
\newcommand{\cI}{{\mathcal I}}
\newcommand{\cK}{{\mathcal K}}
\newcommand{\cL}{{\mathscr L}}
\newcommand{\cN}{{\mathcal N}}
\newcommand{\cU}{{\mathcal U}}
\newcommand{\bone}{\mathbbm{1}}
\newcommand{\suchthat}{:}
\renewcommand{\phi}{\varphi}
\newcommand{\actson}{\mathrel{\curvearrowright}}
\newcommand{\normalsg}{\mathrel{\unlhd}}
\newcommand{\symdiff}{\mathrel{\triangle}}
\newcommand{\algcl}[1]{\overline{#1}}
\renewcommand{\th}[1]{{#1}^{\mathrm{th}}}
\newcommand{\opp}[1]{{#1}^{\mathrm{opp}}}
\newcommand{\red}[1]{{\color{red}{#1}}}

\DeclareMathOperator{\powerset}{{\mathcal{P}}}
\DeclareMathOperator{\im}{im}
\DeclareMathOperator{\id}{id}
\DeclareMathOperator{\End}{End}
\DeclareMathOperator{\LT}{LT}
\DeclareMathOperator{\Aut}{Aut}
\DeclareMathOperator{\Hom}{Hom}
\DeclareMathOperator{\PAut}{PAut}
\DeclareMathOperator{\ch}{char}
\DeclareMathOperator{\curl}{curl}
\renewcommand{\div}{\mathbf{div}}
\newcommand{\Diff}{\text{Diff}}
\newcommand{\FDiff}{\text{FDiff}}
\renewcommand{\bar}{\overline}
\def\p{\partial}
\def\ep{\varepsilon}
\def\st{\;\vert\;}
\newcommand{\abs}[1]{\left\vert #1\right\vert}
\newcommand{\norm}[1]{\left\Vert #1\right\Vert}

\newcommand{\colsection}[1]{
    \begin{beamercolorbox}[colsep*=0ex,dp=2ex,center]{block title}
        \vskip0pt
        \usebeamerfont{block section} #1
        \vskip-1.25ex
        \begin{beamercolorbox}[colsep=0.05ex]{block separator}\end{beamercolorbox}
    \end{beamercolorbox}
    {\parskip0pt\par}
    \vskip0pt
}
%\newenvironment{bblock}[1]{\comment}{\endcomment}
\excludeversion{bblock}

% ====================
% Lengths
% ====================

% If you have N columns, choose \sepwidth and \colwidth such that
% (N+1)*\sepwidth + N*\colwidth = \paperwidth
\newlength{\sepwidth}
\newlength{\colwidth}
\setlength{\sepwidth}{0.020\paperwidth}
\setlength{\colwidth}{0.225\paperwidth}

\newcommand{\separatorcolumn}{\begin{column}{\sepwidth}\end{column}}

% ====================
% Title
% ====================
\title{A Probabilistic Analysis of Enhanced Dissipation}
\author{Ethan Lu}
\institute{Department of Mathematical Sciences, Carnegie Mellon, Pittsburgh}

% ====================
% Footer (optional)
% ====================

\footercontent{
CMU Poster Competition }
% (can be left out to remove footer)

% ====================
% Logo (optional)
% ====================

% use this to include logos on the left and/or right side of the header:
\logoright{\includegraphics[height=7cm]{cmu.png}}
% ====================
% Body
% ====================

\begin{document}

\begin{frame}[t]
    \begin{columns}[t]
        \separatorcolumn

        \begin{column}{\colwidth}
            \colsection{Motivation and Setup}
            \begin{alertblock}{Background/Abstract}
This project concerns itself with the phenomenon of \textbf{enhanced dissipation}, an effect that commonly manifests itself in a variety of physical contexts, ranging from tumor growth and bacterial movement to fluid flow and chemotaxis (through e.g., the Cahn-Hilliard and Keller-Segel equations).
The basic setup can be thought of as follows. Suppose you’re given a container of incompressible fluid and a drop of dye and are tasked with coloring all of the water as quickly as possible.
While you could just drop in the dye and step away, perhaps the more natural thing to do would be to put the dye in and then to stir the fluid, thereby speeding up the process of convection.
This simple observation captures exactly the idea of \emph{enhanced dissipation}, which has become a recent subject of study within the PDE community (see \cite{Thiffeault_2012} for a review).
Although precise formulations of this phenomenon vary significantly from problem to problem, perhaps one of the simplest models to consider is that of the advection–diffusion equation, which reads 
\begin{align}
    \p_t \theta + u \cdot \nabla \theta - \kappa \Delta \theta = 0
\end{align}
where $\theta$ is understood to represent the concentration of a dye, $u$ is a (possibly time-dependent and divergence free) velocity field advecting the fluid, and $\kappa$ is the diffusivity of the dye.
Within this context, a standard measure of dissipation is to consider bounds on the \emph{variance} of such solutions (e.g., bounds on $\norm{ \theta- \int- \theta}_{L^2}$).
Using a simple energy estimate, we can obtain the bound
\[
        \norm{\theta(\cdot,t)}_{L^2_0(\Omega)} \le 
        e^{- \lambda_1 \kappa t}\norm{\theta(\cdot,0)}_{L^2_0(\Omega)} 
\]
for all mean 0 solutions to the equations above, where $\lambda_1$ denotes the smallest non-zero eigenvalue of the Laplacian.
Although such a calculation already yields a useful insight into the nature of the problem, the energy estimate above is inherently limited because it fails to consider advection. Indeed, after integrating by parts, the effect of $u$ cancels, meaning that this bound has no dependence on the bulk motion of the fluid.  
The primary question to answer, then, is to see how different properties of $u$ can be parlayed into improved bounds on the exponent given above, also known as the \emph{dissipation time} of the system.
Towards doing so, the primary relation we will be interested in studying is that between the \emph{ergodicity} of such a system and its dissipation time. %the exact definitions of which will be quantified later. 

            \end{alertblock}

            \begin{block}{Setup and Notation}
                Throughout, our domain of interest will be a fixed smooth Riemannian manifold $M$, which we'll typically assume to just be given by the $n$ dimensional torus, $\mathbb T^n := (\R/\Z)^n$.

                Given any such domain, we can then define the function spaces $\Diff_0(\Omega),\FDiff(\Omega) \subseteq L^2(\Omega;\R^n)$, which are set to be
            \end{block}





        \end{column}
        \separatorcolumn
        \begin{column}{\colwidth}
            \colsection{Continuous Time}
            \begin{block}{Previous Work}
    In continuous time, recent results from \cite{Feng_2019,Zelati_Delgadino_Elgindi_2020} have shown that the naive $\mathcal O( \kappa^{-1})$ bound on the dissipation time can be improved to $\mathcal O(\ln \kappa)^2$ given an ``exponential mixing'' assumption on the underlying flow $u$, the details of which are the focus of this section.
While already a substantial improvement to the naive bounds discussed above, however, there still remains the natural question of whether or not these bounds are optimal.
Heuristically, the answer is no: in addition to intuitive arguments suggesting an optimal bound of order $\mathcal{O}({\ln \kappa})$, several classes of exponentially mixing flows for which this bound holds have been explicitly constructed, suggesting that it could hold in the general case (see \cite{Feng_2019}).

                \begin{thm}
    Suppose that $u$ is a smooth, divergence free, and exponentially mixing velocity field.
    Then for $\kappa \ll 1$, the dissipation time of the associated advection-diffusion equations is bounded by
    \[
        \tau_d \le  C\abs{\ln{\kappa}}^2
    \]
    for some $C > 0$.
                \end{thm}

                In order to prove this result, we will naturally need a number of technical results,which we present now.
            \end{block}


\begin{block}{Technical Results}
\begin{defn}
		The \textbf{dissipation time} associated to the flow generated by $u, \kappa$ is
		\[
			\tau_d := \inf \{t-s \st \norm{\mathcal{S}{s,t}(\theta_s)} \le \dfrac{\norm {\theta_s}}{e} \quad \forall \theta_s \in L^2_0, t \in \R\}.
		\]
	\end{defn}
            \end{block}


            \begin{block}{Sketch of Theorem \#1}
                We now provide a brief sketch of the proof of 
            \end{block}


        \end{column}



        \separatorcolumn

        \begin{column}{\colwidth}
            \colsection{Discrete Time}
            \begin{block}{Motivation and Examples}
                We start with Arnold's original derivation of the Euler equations, noting in particular the relative simplicity of the associated energy.

            \end{block}
            \begin{block}{Markov Processes and Dynamical Systems}

                \begin{proof}[Proof Sketch]
                    We proceed as in the previous result.
                    By first only considering compactly supported velocity fields, we can isolate the contribution of the terms defined on $\Omega$ to deduce the first equation.
                    Considering general velocity fields, combining the Reynolds transport equation and the surface divergence theorem, and doing further computations then yields the other equations.
                \end{proof}

            \end{block}

        \end{column}

        \separatorcolumn

        \begin{column}{\colwidth}
            \vspace{-1.5cm}

            \begin{block} {Proof Sketch}
                Finally, we introduce another degree of freedom by allowing the surfactants to move along the boundary independently of the mass in the interior, as parameterized by a function $\beta$.
                \begin{thm}
Let $M$ be a Riemannian manifold and $T: M \to M$ be a smooth exponentially mixing dynamical system (in the sense of Bernoulli systems).
			Suppose further that $K_\kappa$ is a homogenous Markov transition kernel on $M$.
			Then the mixing time of the Markov process associated to $T \star K_\kappa$ satisfies the estimate
			\[
				\tau_{mix} \le C\abs{\ln \kappa}.
			\]

                \end{thm}


                Note that while this result and the previous result appear to be very similar, the surfactant velocity $u_s$ is now generated not only by the motion of $\eta$ but also by the motion of the surfactants $\beta$, which complicates the corresponding terms. %while $u$ depends solely upon $\eta$.
            \end{block}

            %\begin{block}{Acknowledgements}
            %    Thanks 
                %This project was joint work with Jonathan Jenkins and Desmond Reed.
                %We would also like to thank advisor Ian Tice for his guidance and mentorship throughout this project. This project was supported by funding from the NSF CAREER grant (\texttt{\#}1653161).
            %\end{block}
            \begin{block}{References}
                \vspace{-0.3em}
                {
                    \small
                    %\tiny
                    \nocite{*}
                    \begin{multicols}{2}
                        \bibliographystyle{abbrv}
                        \bibliography{refs}
                \end{multicols}}
            \end{block}
        \end{column}
        \separatorcolumn
    \end{columns}
\end{frame}

\end{document}
